\section{Algorithmic Details}

The goal of the {\it extract concurrent} refactoring is to allow
users to (somewhat) safely introduce parallelism/concurrency to
array accesses and updates that are done within a {\tt for}
loop.
%Since so much work has been done attempting to automate the
%parallel transformation of such loops, and
Since loops tend to be execution hotspots, and X10 supports explicit
asynchronous event constructs, this
refactoring is the type that X10 users should expect to find in an IDE
like Eclipse.

The general schema for the transformation is illustrated by the two
code skeletons shown in Figure~\ref{fig:transformSchema}, which
depict the original and the transformed code.

\begin{figure}[tp]
 Before:
\begin{code}
for(\=$\ldots$) \{\\
\> $\ldots$ target $\ldots$\\
\}
\end{code}

After:
\begin{code}
for(\=$\ldots$)\= \{\\
\> $\ldots$\\
\> targetFuture = future \{ target \};\\
\> $\ldots$\\
\}\\\\
for($\ldots$) \{\\
\> $\ldots$ targetFuture.force() $\ldots$\\
\}
\end{code}
\caption{\label{fig:transformSchema}The transformation schema for futurizing an expression.}
\end{figure}

A basic assumption of this transformation is that it will introduce
no activities that remain unfinished outside the scope of the parent
construct containing the target distributed array.
As a result, the developer need not expend mental effort to determine
when the results of the activity will be available.
%At the same time, in the {\tt async} case, the code generated will not
%allow overlapped execution between the two loops.
%This is conservatively safe, but comes at the cost of potentially
%useful concurrency.
In this case, overlap between the two loops can be
easily permitted, since the {\tt force()} calls in the second loop
perform all of the necessary synchronization.
%The synchronization required to in{\tt async} case is
%significantly more complex, due to the difference in nature between the
%data-centric {\tt future} construct and the operation-centric {\tt async}.

\subsection{Pre-conditions}

\begin{figure}[tp]
\begin{code}
Lo\=op (InductionVars $\ldots$)\{ \\
\> Block1 \\
\> Result = (Exp1 op f(Target[])) op Exp2; \\
\> Block2 \\
\}
\end{code}
\caption{\label{fig:basic_loop}The generic loop structure for the {\it
extract concurrent} refactoring.}
\end{figure}

In general, the code for a candidate {\tt for} loop will look as in
Figure~\ref{fig:basic_loop}. In this case, the user would be
interested in potentially making the function {\tt f(Target[])} representing
the targeted expression -- which is not
necessarily a method call -- into a futurized expression. The
candidate expression or statement will be called the {\em target}. Here,
{\tt Block1} and {\tt Block2} are both arbitrary blocks of
code%, although they must not define an {\tt atomic} block that starts
%in {\tt Block1} and ends in {\tt Block2}; concurrent activities cannot be
%spawned from within an {\tt atomic} block. There may also be no clock
%manipulations present in the body of the loop -- the sole exception
%here is {\tt resume}, since it is a non-blocking update to the clock
%which cannot have loop-carried dependencies.

As with all parallelizing algorithms, the most important information to
track is loop-carried dependencies. In this case, all loop-carried dependencies
must be reproducible on all executions of the loop. As a result, the following
pre-conditions must be met in order to refactor the code:
\begin{itemize}
\item No loop-carried dependencies are allowed on the target distributed array.
%\item The pivot must not have side-effects that affect any loop-carried
%      dependencies.
\item The target must not produce side-effects.
\item There must be no loop-carried dependencies on user input or the
      evaluation of any chaotic functions (e.g., random number generators).
%\item {\tt Exp2} and {\tt Block2} must not have side-effects on the target
\item Loop-carried dependent statements that occur semantically after the
target may not have side-effects that directly affect the target.
\end{itemize}

\subsection{Interactions with X10 Constructs}

In this section, the interaction of the transformation with various
X10 constructs will be discussed. Because the transformation is
aimed at targets residing within loops, any errors that are a
result of poorly formed code outside of the loop will be omitted from
this discussion. It is noted, however, that any asynchronous activity
that remains unfinished or unforced before the loop is entered may
create data races within the program.

{\bf Asyncs} -- %If the target is contained within an unfinished async
%block, 
%it is not clear that the transformation will provide improved
%parallelism unless the activity occurs in a different place than the
%target. Such a situation should be 
%the decision to apply the refactoring is left to the programmer
%to
%determine, although support could be provided to warn a programmer who
%ttempts to refactor under these circumstances. 
Unfinished
asynchronous activities that have (possibly indirect) side-effects on
the target and occur semantically before the target in the loop should be
must stop the transformation from occurring.
%This points out a potential race condition in the original 
%code, so we abort the transformation to avoid exacerbating the problem.
All other finished
asynchronous activity is allowed if it conforms with the X10 language
rules and meets the pre-conditions stated
in the previous subsection. Note that choosing to extract concurrent on
a target within a finished async block effectively precludes
introduction of new parallelism.

{\bf Futures} -- Futures, in general, are treated the same as asyncs.
However, code inserted between the two
constructed loops must not have side-effects on any futurized expression that
is used in the second loop

 {\bf Atomics} -- As long as the target does not fall
within an atomic block, there are no restrictions on the use of atomic in the
loop. The target may contain an atomic block itself.

{\bf Clocks} -- Because the transformation creates new paths in the program,
%many clock-associated operations are disallowed within the loop. In
%particular, the {\tt registered} and {\tt resume} methods may be called at any
%time within the loop. But 
{\tt next} and {\tt drop} may not be called on a clock in the
first half of the loop or in any statement which has loop-carried
dependencies. %Since clock operations that exist in the loop probably
%signal that a full iteration of a loop has occurred, executing one of
%these operations during evaluation of the first half of the transformed loop
%would break this assumption. However, {\tt registered} and {\tt resume} are
%harmless operations in that multiple calls to either will not cause any
%effects when {\tt next} and {\tt drop} cannot be called. Since {\tt resume}
%does not return a value, it is also safe to completely remove it from the
%first loop and place it in the second with a slicing algorithm. 
After execution of the target can be guaranteed to have completed (i.e., at any
non-loop-carried dependent statement after the target), it is safe to perform
any of the clock operations.

{\bf Exceptions} -- If the target is within a try block, then as a
conservative measure, we force the transformation to fail.
This is based on the semantics of X10: uncaught exceptions thrown by
asynchronous activities are accumulated by the collecting {\tt finish}.
Thus, exceptions thrown by activities spawned inside a try block are not
propagated to the try block's catch clauses.
%In particular, if the exception's purpose was to short-circuit the loop,
%delaying its delivery would result in a change to the loop's control
%flow (i.e., in improper execution of code that would not otherwise
%have executed).
%Note that forcing the transformation to fail is a bit overly conservative in
%that the target may not throw any exception named in the {\tt catch} blocks,
%so that moving the target would not change the exception behavior of
%the code.