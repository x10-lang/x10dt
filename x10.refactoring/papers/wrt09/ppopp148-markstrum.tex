\documentclass[natbib]{sigplanconf}
%\documentclass{llncs}
\usepackage{times}
\usepackage{ifthen}
\usepackage{bcprules}
\usepackage{listings}
\usepackage{url}

\newenvironment{code}{\begin{center}\begin{small}\begin{tt}\begin{tabbing}}
{\end{tabbing}\end{tt}\end{small}\end{center}}

\newcommand{\subf}[2]{\begin{minipage}[t]{3in} #2 \end{minipage}\vspace{-.1in}\\\begin{minipage}[t]{3in}\center (#1) \end{minipage}\\}

\newcommand{\twocolfig}[2]{\begin{minipage}[t]{3in} #1 \end{minipage}\hspace{.5in}\begin{minipage}[t]{3in} #2 \end{minipage}}

\begin{document}

\special{papersize=8.5in,11in}
\setlength{\pdfpageheight}{\paperheight}
\setlength{\pdfpagewidth}{\paperwidth}

\authorpermission
\conferenceinfo{PPoPP'09,} {February 14--18, 2009, Raleigh, North Carolina, USA.}
\CopyrightYear{2009}
\copyrightdata{978-1-60558-397-6/09/02}
\title{Towards Concurrency Refactoring for X10}

\authorinfo{Shane Markstrum}{University of California, Los
Angeles\\Computer Science Department\\Los Angeles, CA USA}{smarkstr@cs.ucla.edu}
\authorinfo{Robert M. Fuhrer}{IBM T.J. Watson Research
Center\\Hawthorne, NY USA}{rfuhrer@us.ibm.com}
\authorinfo{Todd Millstein}{University of California, Los Angeles\\Computer
Science Department\\Los Angeles, CA USA}{todd@cs.ucla.edu}

\maketitle

\begin{abstract}
In this poster, we present our vision of refactoring 
support for languages with a partitioned global address space memory model as 
embodied in the X10 programming language. We examine a novel refactoring, 
extract concurrent, that introduces additional concurrency within a loop by
arranging for some user-selected code in the loop body to run in parallel
with other iterations of the loop. We discuss the mechanisms and challenges 
for implementing this refactoring and how development of this refactoring 
provides insight for designing future refactorings.
\end{abstract}

\category{D.1.3}{Programming Techniques}{Concurrent Programming}
\category{D.2.6}{Software Engineering}{Programming Environments}
\category{F.3.2}{Logics and Meanings of Programs}{Semantics of
Programming Languages}[Program Analysis]

\terms
Algorithms, Languages


\section{Introduction}

The industry's shift to multicore processors has sparked a trend
toward pushing parallel programming into the mainstream.  This trend
poses a significant challenge, since creating and maintaining parallel
programs that are both efficient and reliable is notoriously
difficult.  One response to this challenge
by the programming languages community has been to create new (and
revisit old) language
abstractions and programming models 
for parallel programming and to
develop new languages based on these abstractions.

While new languages can greatly aid programmers in developing 
parallel programs, we believe that new languages cannot
achieve mainstream success without associated tooling support.
Modern integrated development environments
(IDEs) such as Eclipse~\cite{eclipse} provide many benefits that programmers
have come to rely upon, helping
them to more easily navigate through a program, understand
dependencies among parts of a program, and safely evolve a program to
improve its quality along some dimension.  The latter benefit is typically
provided through code {\em refactorings}.  

We believe that
specialized refactoring support will be a critical tool to help programmers improve the
quality of parallel code.  Such refactorings could be used
to improve efficiency while preserving program behavior and key
concurrency invariants (e.g., atomicity, deadlock-freedom).

Any set of refactorings will of necessity be tailored to the needs of a particular
parallel programming model and language.
We focus on providing refactoring support for the
\emph{partitioned global address space} (PGAS) memory model as
embodied in the X10~\cite{X10,Charles05}, UPC~\cite{ElGhazawi03} and
Titanium~\cite{Yelick98} programming languages.  In this model, the
programmer sees a uniform 
representation of data and data structures over distributed nodes, regardless
of the physical location of the data.  However, each piece of data is
assigned to a fixed partition (or {\em place} in X10 terminology)
and can only be accessed by {\em activities}
that run at that place.

In this presentation, we focus on the X10 language, an object-oriented language
providing first-class, high-level constructs for asynchronous activities, synchronization,
phased computations, data distribution, and atomicity.
By incorporating such abstractions as first-class constructs in the language,
the burden of reasoning about various program properties is often reduced
from global reasoning involving complex control flow to simple and modular
reasoning about lexical containment.
In fact, many interesting properties (e.g. deadlock freedom) can often
be ensured statically.
In this way, X10's constructs simplify both the programmer's task of
understanding and tools' tasks of static program analysis.

The PGAS model provides particular opportunities and challenges for
automated refactorings.  On the one hand, code transformations are
simplified since the code need not explicitly handle inter-processor
communication.  
On the other hand, transformations must properly handle the
asynchronicity that arises among activities and must respect
the synchronization constraints imposed on these activities by the
semantics of the various language constructs. 
In this paper we describe an initial concurrency
refactoring for X10 that we
have been developing inside the X10DT plugin for Eclipse.

\section{Extract Concurrent}

As a first step toward our vision, we are developing a refactoring
called {\em extract
concurrent} for the X10 language within the X10DT, our Eclipse-based IDE. 
The transformation introduces concurrency within a loop by arranging
for some user-selected code in the loop body 
to run in parallel with other iterations of the 
loop.

As an example, consider the X10 code in Figure~\ref{fig:CHM-X10}, an excerpt from
an X10 implementation of the {\tt ConcurrentHashMap} class from the Java 
standard utilities. In this snippet from the {\tt containsValue} method,
a snapshot of the modification status of each map segment is taken before
determining whether a particular segment contains the desired value. This
approach allows value lookup to occur without locking the entire map. Given
the PGAS model for data distribution, the individual elements of the array 
{\tt segments} could reside anywhere in the global address space, increasing
their access cost. Further, each call to {\tt modCount()} must block until the
call completes, per X10 semantics. Thus it is possible that asynchronously
executing the {\tt modCount} method for each array element will speed up the
overall execution of the loop.

\begin{figure}[tp]
  \begin{code}
int mcsum=0; \\
fo\=r ( point p : segments ) \{ \\
\>  mc[p] = segments[p].modCount(); \\
\>  mcsum += mc[p]; \\
\>  if\=(segments[p].containsValue(value)) \\
\>\>    return true; \\
\}
  \end{code}
\caption{\label{fig:CHM-X10} An excerpt from the X10 version of the Java 
library {\tt java.util.concurrent.ConcurrentHashMap}.
}
\end{figure}

One way to introduce this concurrency, shown in Figure~\ref{fig:CHM-X10-future},
is to execute each of the {\tt modCount} invocations as {\tt future}s in a new loop
and only synchronize with those executions via a {\tt force}
operation when their results are actually needed.
This is in fact the transformation that was manually applied to
the code in our example after the initial translation
from Java to X10.  Our
{\em extract concurrent} refactoring automates this transformation and
ensures that the transformation preserves program behavior.  Our
refactoring also supports a generalization of this transformation that
allows a block of statements to be safely executed asynchronously,
but we do not discuss it here for brevity's sake.

\begin{figure}[tp]
  \begin{code}
int mcsum=0; \\
future<int>[.] f\_\=segments = \\
\>new future<int>[segments.region]; \\
fo\=r ( point p : segments ) \{ \\
\>  f\_\=segments[p] = \\
\>\>future(segments[p])\{segments[p].modCount()\}; \\
\} \\ \\
for ( point p : segments ) \{ \\
\>  mc[p] = f\_segments[p].force(); \\
\>  mcsum += mc[p]; \\
\>  if\=(segments[p].containsValue(value)) \\
\>\>    return true; \\
\} 
  \end{code}
\caption{\label{fig:CHM-X10-future} A transformation of the program excerpt 
from Figure~\ref{fig:CHM-X10} introducing additional concurrency via the 
X10 {\tt future} construct.}
\end{figure}

Our refactoring involves two main components:
\begin{enumerate}
\item {\em Loop dependence analysis.} Since introducing parallelism in
the middle of a loop might affect the ability of other statements in a
loop to evaluate properly, it is important that loops do not depend
on the results of any asynchronously executed statements. We
have developed a set of analyses to determine whether {\em extract
concurrent} will adversely affect the execution of the code and
violate its perceived sequential consistency.

\item {\em Transformation pattern.} We have developed a general
pattern for the {\em extract concurrent} transformation on viable
sequential loops.
The pattern splits the loop in two, as shown in Figure~\ref{fig:CHM-X10-future}: the
first loop introduces the desired statement- and/or expression-level
parallelism, while the second loop synchronizes with and utilizes the
results of the asynchronous execution.
\end{enumerate}

The concurrency
constructs and the PGAS model used by X10 simplify the analysis required to determine
when program transformations are safe. For example, the static analysis required to
determine whether a statement may be executed asynchronously is
reduced to determining local and loop-carried dependencies that prevent a
statement from being asynchronously executed. The example highlights a perceived benefit of
refactoring X10: data locality and asynchronous execution are separable, or
{\em lateral}, concerns. Thus, a programmer may manipulate either the amount of
program concurrency or the data distribution while keeping the other relatively
fixed.

We have built a prototype of the {\em extract concurrent} 
refactoring and the supporting analysis
in the X10DT, and we are in the process of refining the
implementation and experimenting with it on some X10 applications. 
Although the present implementation targets X10, we believe that the
transformation is readily applicable to other languages with a PGAS
programming model.

\section{Related Work}
\label{sec:related}

To our knowledge, ours is the first work to consider automated code
refactorings in the context of the PGAS model.  However,
a number of IDEs and
tools have been developed to aid parallel language users. 
We elide discussion here of automatic parallelization
techniques and parallel program analysis to focus on related work in tooling
support.

The SUIF Explorer~\cite{Liao99} assists programmer parallelizing of
code by providing a dynamic dependence analyzer, but does not feature
integrated refactoring support. The ParaScope Editor~\cite{Kennedy91} is an IDE that
enables exploration and manipulation of
loop-level parallelism in a Fortran-like language. It makes
analysis results and a number of program transformations, including {\em loop
interchange} and {\em loop distribution}, available to the user.

Photran~\cite{Overbey05} is an IDE that intends to, but does not yet, 
provide concurrency refactoring support for HPC applications in Fortran.
TSF~\cite{TSF} is an IDE tool for writing
transformation scripts and transforming parallel Fortran programs. Some of
the transformations it provides have preconditions for verifying soundness,
which is a feature we also integrate into the extract concurrent
refactoring.

\bibliographystyle{plainnat}
\begin{thebibliography}{99}

\bibitem{TSF}
F.~Bodin, Y.~M{\'e}vel, and R.~Quiniou.
\newblock A user level program transformation tool.
\newblock In {\em International Conference on Supercomputing}, pages 180--187,
  1998.

\bibitem{Charles05}
P.~Charles, C.~Grothoff, V.~Saraswat, C.~Donawa, A.~Kielstra, K.~Ebcioglu,
  C.~von Praun, and V.~Sarkar.
\newblock X10: an object-oriented approach to non-uniform cluster computing.
\newblock In {\em OOPSLA '05: Proceedings of the 20th annual ACM SIGPLAN
  conference on Object oriented programming, systems, languages, and
  applications}, pages 519--538, New York, NY, USA, 2005. ACM Press.

\bibitem{eclipse}
{Eclipse} home page.
\newblock \url{http://www.eclipse.org}.

\bibitem{ElGhazawi03}
T.~A. El-Ghazawi, W.~W. Carlson, and J.~M. Draper.
\newblock {U}{P}{C} language specifications v1.1.1, October 2003.

\bibitem{Kennedy91}
K.~Kennedy, K.~S. M\raisebox{.2em}{c}Kinley, and C.-W. Tseng.
\newblock Analysis and transformation in the {Para{Scope} {Editor}}.
\newblock In {\em Proceedings of the 1991 {ACM} International Conference on
  Supercomputing}, Cologne, Germany, 1991.

\bibitem{Liao99}
S.-W. Liao, A.~Diwan, J.~Robert P.~Bosch, A.~Ghuloum, and M.~S. Lam.
\newblock {S}{U}{I}{F} {Explorer}: An interactive and interprocedural
  parallelizer.
\newblock In {\em PPoPP '99: Proceedings of the Seventh ACM SIGPLAN Symposium
  on Principles and Practice of Parallel Programming}, pages 37--48, New York,
  NY, USA, 1999. ACM Press.

\bibitem{Overbey05}
J.~Overbey, S.~Xanthos, R.~Johnson, and B.~Foote.
\newblock Refactorings for {Fortran} and high-performance computing.
\newblock In {\em SE-HPCS '05: Proceedings of the 2nd International Workshop on
  Software Engineering for High Performance Computing System Applications},
  pages 37--39, New York, NY, USA, 2005. ACM Press.

\bibitem{X10}
V.~Saraswat.
\newblock Report on the experimental language {X10} v0.41.
\newblock http://www.research.ibm.com/x10/.

\bibitem{Yelick98}
K.~Yelick, L.~Semenzato, G.~Pike, C.~Miyamoto, B.~Liblit, A.~Krishnamurthy,
  P.~Hilfinger, S.~Graham, D.~Gay, P.~Colella, and A.~Aiken.
\newblock {Titanium}: {A} high-performance {Java} dialect.
\newblock In {ACM}, editor, {\em {ACM} 1998 Workshop on Java for
  High-Performance Network Computing}, New York, NY 10036, USA, 1998. ACM
  Press.

\end{thebibliography}

\end{document}
